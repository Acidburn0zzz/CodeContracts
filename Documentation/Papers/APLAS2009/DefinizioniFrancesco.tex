% Definizione dei simboli

\newcommand{\eg}{\textit{e.g.}}
\newcommand{\ie}{\textit{i.e.}}

% Definizione dell' operatore di astrazione
\newcommand{\abs}[1]{\ensuremath{\bar{\mathsf{#1}}}}

\newcommand{\sx}{\llbracket}
\newcommand{\dx}{\rrbracket}

  % Semantica generica
\newcommand{\sem}[1]{\ensuremath{\sx \mathtt{#1} \dx}}

  % Semantica che prende un nome
\newcommand{\semantica}[2]{\ensuremath{\mathbb{#1}\sem{#2}}}

\newcommand{\BigSemantica}[2]{\ensuremath{\mathbb{#1}{\Bigg\llbracket #2 \Bigg\rrbracket}}}

  % Semantiche generiche
\newcommand{\csem}[1]{\ensuremath{\semantica{s}{#1}}}
\newcommand{\asem}[1]{\ensuremath{\semantica{\abs{s}}{#1}}}
\newcommand{\asemRefined}[1]{\ensuremath{\semantica{\abs{s}^*}{#1}}}

  % Funzione di ``transfer''
\newcommand{\trasf}[1]{\ensuremath{\semantica{t}{#1}}}
\newcommand{\atrasf}[1]{\ensuremath{\semantica{\abs{t}}{#1}}}

  % Semantica di: costruttore,  metodo
\newcommand{\semcostr}[1]{\semantica{i}{#1}}
\newcommand{\semmetodo}[1]{\semantica{m}{#1}}
\newcommand{\semoggetto}[1]{\semantica{o}{#1}}
\newcommand{\semclasse}[1]{\semantica{c}{#1}}

  % Semantica Colleting Traces
\newcommand{\Semmetodo}[1]{\semantica{M}{#1}}
\newcommand{\Semoggetto}[1]{\semantica{O}{#1}}
\newcommand{\Semclasse}[1]{\semantica{C}{#1}}

  % Semantics Reachable States
\newcommand{\rsemcostr}[1]{\semantica{I}{#1}}
\newcommand{\rsemmetodo}[1]{\Semmetodo{#1}}
\newcommand{\rsemoggetto}[1]{\semantica{O}{#1}}
\newcommand{\rsemclasse}[1]{\semantica{C}{#1}}

  % Seamntica Astratta
\newcommand{\asemcostr}[1]{\semantica{\bar{I}}{#1}}
\newcommand{\asemmetodo}[1]{\semantica{\bar{M}}{#1}}
\newcommand{\asemoggetto}[1]{\semantica{\bar{O}}{#1}}
\newcommand{\asemclasse}[1]{\semantica{\bar{C}}{#1}}

  % Operazioni su stati
\newcommand{\less}{\ensuremath{\sqsubseteq}}
\newcommand{\join}{\ensuremath{\sqcup}}
\newcommand{\meet}{\ensuremath{\sqcap}}
\newcommand{\bottom}{\ensuremath{\bot}}
\newcommand{\bigjoin}{\ensuremath{\bigsqcup}}

  % Semantica astratta
\newcommand{\asemantica}[2]{\semantica{\bar{#1}}{#2}}
\newcommand{\asemanticaBig}[2]{\BigSemantica{\bar{#1}}{#2}}


  % Semantica astratta BIS, per aggiungere pedici
\newcommand{\asemanticaBis}[3]{\semantica{\bar{#1}_{\textit{#2}}}{#3}}

  % Operazioni astratte
\newcommand{\nabs}[1]{\bar{#1}}
\newcommand{\abot}{\nabs{\bot}}
\newcommand{\atp}{\nabs{\ensuremath{\top}}}
\newcommand{\aless}{\nabs{\less}}
\newcommand{\asup}{\ensuremath{\nabs{\sqsupseteq}}}
\newcommand{\acup}{\ensuremath{\nabs{\sqcup}}}
\newcommand{\ajoin}{\acup}
\newcommand{\abigcup}[1]{\nabs{\bigsqcup_{#1}}}
\newcommand{\abigjoin}[1]{\abigcup{#1}}
\newcommand{\acap}{\ensuremath{\nabs{\sqcap}}}
\newcommand{\ameet}{\acap}
\newcommand{\abigcap}[1]{\nabs{\bigsqcap_{#1}}}
\newcommand{\abigmeet}[1]{\abigcap{#1}}


 % lfp : least fixpoint
\newcommand{\lfp}[2]{\ensuremath{\mathrm{lfp}^{#1}_{#2}}}

 % gfp: greatest fixpoint
\newcommand{\gfp}[2]{\ensuremath{\mathrm{gfp}^{#1}_{#2}}}


% Insiemi e Funzioni

  % Insieme, elemento di insieme, nessun elemento
\newcommand{\insieme}[1]{\ensuremath{\mathsf{#1}}}
\newcommand{\el}[1]{\ensuremath{\mathsf{#1}}}
\newcommand{\noel}{\el{\bot}}

  % Insieme delle parti
\newcommand{\parti}[1]{\ensuremath{\mathcal{P}{(#1)}}}
\newcommand{\partiFin}[1]{\ensuremath{\mathcal{P}_{\mathrm{fin}}{(#1)}}}


  % Elementi di insiemi
\newcommand{\noval}{\ensuremath{\baro}}

  % Cardinalita di inisiemi
\newcommand{\cardinalita}[1]{\ensuremath{\# #1}}

  % Env, Store, Addr, Val, Label : Scorciatoie
\newcommand{\env}{\insieme{Env}}
\newcommand{\Env}{\env}

\newcommand{\aenv}{\ensuremath{\bar{\env}}}

\newcommand{\store}{\insieme{Store}}
\newcommand{\Store}{\store}

\newcommand{\astore}{\ensuremath{\overline{\store}}}

\newcommand{\AState}{\abs{\Sigma}}
\newcommand{\astate}{\AState}

\newcommand{\addr}{\insieme{Addr}}
\newcommand{\Addr}{\addr}

\newcommand{\aaddr}{\ensuremath{\overline{\addr}}}

\newcommand{\Reference}{\insieme{Ref}}
\newcommand{\reference}{\reference}

\newcommand{\AReference}{\ensuremath{\overline{\Reference}}}

\newcommand{\val}{\insieme{Val}}

\newcommand{\etichetta}{\insieme{Label}}
\newcommand{\cont}{\ensuremath{\kappa}}

  % Stati
\newcommand{\stati}[1]{\ensuremath{\Sigma}}
  % Tracce
\newcommand{\tracce}[1]{\ensuremath{\mathcal{T}(#1)}}
  % Transizione
\newcommand{\tr}[1]{\ensuremath{\xrightarrow{#1}}}

  % Insieme di funzioni
\newcommand{\funzione}[2]{\ensuremath{[#1 \rightarrow#2]}}

  % Funzione vuota
\newcommand{\emptyfun}{\ensuremath{[\cdot]}}

  % Insieme di funzioni monotone e joim-morphism
\newcommand{\funmon}[2]{\ensuremath{[#1 \stackrel{m}{\longrightarrow} #2]}}
\newcommand{\funjoin}[2]{\ensuremath{[#1 \stackrel{\sqcap}{\longrightarrow} #2]}}

% Astratto
  % Dominio concreto
\newcommand{\dom}[1]{\insieme{#1}}

  % Dominio astratto
\newcommand{\adom}[1]{\ensuremath{{\mathsf{#1}}}}

 % Vari domini
\newcommand{\Intervals}{\dom{Intv}}
\newcommand{\LT}{\dom{LT}}
\newcommand{\Boxes}{\dom{Boxes}}
\newcommand{\DBM}{\dom{DBM}}
\newcommand{\Octagons}{\dom{Oct}}
\newcommand{\Polyhedra}{\dom{Poly}}
\newcommand{\Poly}{\Polyhedra}
\newcommand{\Lineq}{\dom{LinEq}}
\newcommand{\LinEq}{\Lineq}
\newcommand{\Karr}{\Lineq}
\newcommand{\Symbolic}{\dom{Symb}}
\newcommand{\Pentagons}{\dom{Pnt}}
\newcommand{\SubPolyhedra}{\dom{SubPoly}}
\newcommand{\SubPoly}{\SubPolyhedra}
\newcommand{\Subpoly}{\SubPolyhedra}

  % Elemento di dominio astratto
\newcommand{\ael}[1]{\ensuremath{\abs{\el{#1}}}}
\newcommand{\aelBis}[2]{\ensuremath{\abs{\el{#1}}_\el{#2}}}

  % Funzione di astrazione e concretizzazione
\newcommand{\alfa}{\ensuremath{\alpha}}
%\renewcommand{\gamma}{\ensuremath{\gamma}}
\newcommand{\gm}{\ensuremath{\gamma}}


% Funzioni usate nella tesi
  % Transizione basica
\newcommand{\prossimo}{\ensuremath{\mathrm{next}}}
%\newcommand{\next}{\ensuremath{\mathrm{next}}}
\newcommand{\nextDir}{\ensuremath{\mathrm{next_{dir}}}}
\newcommand{\nextInd}{\ensuremath{\mathrm{next_{ind}}}}
\newcommand{\prossimoDir}{\nextDir}
\newcommand{\prossimoInd}{\nextInd}

  % Transizione ``collecting''
\newcommand{\Next}{\ensuremath{\mathrm{Next}}}
\newcommand{\NextDir}{\ensuremath{\mathrm{Next_{dir}}}}
\newcommand{\NextInd}{\ensuremath{\mathrm{Next_{ind}}}}

  % Reachable addresses
\newcommand{\reachable}{\ensuremath{\mathrm{reachable}}}
  % Update function
\newcommand{\update}{\ensuremath{\mathrm{update}}}
  % Storia delle interazioni
%\newcommand{\etichette}{\ensuremath{\mathrm{labels}}}
\newcommand{\storia}{\ensuremath{\mathrm{history}}}

  % Contesto
\newcommand{\contesto}{\ensuremath{\mathrm{Context}}}

  % Contesto Astratto
\newcommand{\acontesto}{\ensuremath{\mathrm{\overline{Context}}}}

% Tracce : definizioni
\newcommand{\stringavuota}{\ensuremath{\epsilon}}


% Funzioni di astrazione
  % Prima Astrazione, quella dei metodi
\newcommand{\alfaPrima}{\ensuremath{\alpha_{\Yright}}}
\newcommand{\gammaPrima}{\ensuremath{\gamma_{\Yright}}}
  % Versione higher-order
\newcommand{\alfaPrimaDot}{\ensuremath{\dot{\alpha}_{\Yright}}}

  % Seconda Astrazione, quella degli stati
\newcommand{\alfaSeconda}{\ensuremath{\alpha_{\circ}}}
\newcommand{\gammaSeconda}{\ensuremath{\gamma_{\circ}}}
  % Versione higher-order
\newcommand{\alfaSecondaDot}{\ensuremath{\dot{\alpha}_{\circ}}}


\newcommand{\alfaAddr}{\ensuremath{\alpha_\el{a}}}
\newcommand{\gammaAddr}{\ensuremath{\gamma_\el{a}}}

\newcommand{\gammaBool}{\ensuremath{\gamma_\el{b}}}

\newcommand{\gammaEnv}{\ensuremath{\gamma_\el{e}}}

\newcommand{\gammaOct}{\ensuremath{\gamma_\el{o}}}
\newcommand{\gammaDOct}{\ensuremath{\gamma_\el{do}}}

\newcommand{\gammaRef}{\ensuremath{\gamma_\el{r}}}

\newcommand{\gammaStore}{\ensuremath{\gamma_\el{s}}}

\newcommand{\gammaState}{\ensuremath{\gamma_{\astate}}}

% Estensioni per OO

  % Classe
\newcommand{\classe}[1]{\ensuremath{\mathtt{#1}}}
\newcommand{\classi}{\ensuremath{\mathcal{C}}}

  % Gerarchia
\newcommand{\gerarchia}[1]{\ensuremath{\mathcal{#1}}}
\newcommand{\gerarchie}{\ensuremath{\mathbb{H}}}

  % Operatori su Gerarchie 
\newcommand{\inserisciClasse}{\ensuremath{\uplus}}
\newcommand{\costruisciClasse}{\ensuremath{\beta}}
\newcommand{\unisciGerarchie}{\ensuremath{\Cup}}

\newcommand{\aclasse}[1]{\ensuremath{\abs{\mathtt{#1}}}}
\newcommand{\aclasseLong}[1]{\ensuremath{\overline{\mathtt{#1}}}}
  % Oggetto, instanza di classe
\newcommand{\oggetto}[1]{\ensuremath{\mathsf{#1}}}
  % Metodo
\newcommand{\metodo}[1]{\ensuremath{\mathtt{#1}}}
  % Campo
\newcommand{\campo}[1]{\ensuremath{\mathtt{#1}}}

  % Astrazioni di campi
\newcommand{\acampo}[1]{\ensuremath{\mathtt{\abs{#1}}}}

  % Astrazione di metodi con constraints
\newcommand{\ametodo}[1]{\ensuremath{\mathtt{\abs{#1}}}}

% Dominio delle Tracce Collecting

\newcommand{\adomTracce}{\ensuremath{\adom{D}_{\Yright}}}

\newcommand{\atopTracce}{\ensuremath{\abs{\top}_{\Yright}}}
\newcommand{\abottomTracce}{\ensuremath{\abs{\bot}_{\Yright}}}

\newcommand{\alessTracce}{\ensuremath{\abs{\subseteq}_{\Yright}}}
\newcommand{\alessTraccia}{\ensuremath{\abs{\subseteq}^{\tau}_{\Yright}}}

\newcommand{\ajoinTracce}{\ensuremath{\abs{\cup}_{\Yright}}}
\newcommand{\ajoinTraccia}{\ensuremath{\abs{\cup}^{\tau}_{\Yright}}}

\newcommand{\ameetTracce}{\ensuremath{\abs{\cap}_{\Yright}}}
\newcommand{\ameetTraccia}{\ensuremath{\abs{\cap}^{\tau}_{\Yright}}}

% Scorciatoie

  % ``Prende''
\newcommand{\prende}{\ensuremath{\mapsto}}

  % Simbolo di definizione
\newcommand{\df}{\ensuremath{\triangleq}}

  % Tupla
\newcommand{\tupla}[1]{\ensuremath{\langle #1 \rangle}}

  % Stati Iniziali
\newcommand{\statiiniziali}[1]{\ensuremath{S_0\tupla{#1}}}

  % Modularita
\newcommand{\hasA}{has-A}
\newcommand{\isA}{is-A}

  % Riferimento ad una formula
\newcommand{\formula}[1]{\ensuremath{\mathrm{(\ref{#1})}}}

% Complessita

  % O-notation
\newcommand{\costo}[1]{\ensuremath{\kappa_{#1}}}


% Definizioni per i domini di vincoli

\newcommand{\Vars}{\ensuremath{\mathtt{Vars}}}
\newcommand{\C}{\ensuremath{\dom{C}}}
\newcommand{\initC}{\ensuremath{\mathrm{initial_{\C}}}}

\newcommand{\dominio}{\ensuremath{\mathrm{dom}}}
\newcommand{\range}{\ensuremath{\mathrm{range}}}

  % Relazioni
\newcommand{\rel}[1]{\ensuremath{\mathsf{\rho}[\mathtt{#1}]}}
\newcommand{\conDuepar}[2]{\ensuremath{\con'[\mathtt{#1}, \mathtt{#2}]}}
\newcommand{\conTrepar}[3]{\ensuremath{\con[\mathtt{#1}, \allowbreak \mathtt{#2}, \allowbreak \mathtt{#3}]}}
\newcommand{\relass}[1]{\ensuremath{\rho(#1)}}

\newcommand{\rgamma}{\ensuremath{\gamma_{\rho}}}
\newcommand{\cgamma}{\ensuremath{\gamma_{\C}}}

 % Lettera
\newcommand{\con}{\ensuremath{\mathsf{c}}}

 % Sequenza
\newcommand{\seq}[1]{\ensuremath{\vartriangleright_\mathtt{#1}}}


 % Semantiche per i vincoli
\newcommand{\semc}[1]{\semantica{s}{#1}}

\newcommand{\semtracce}[1]{\semantica{t}{#1}}
\newcommand{\asemtracce}[1]{\semantica{\bar{t}}{#1}}
\newcommand{\asemtracces}[1]{\semantica{\bar{s}}{#1}}


\newcommand{\modsem}[1]{\ensuremath{\semantica{T}{#1}}}
\newcommand{\amodsem}[1]{\semantica{\bar{T}}{#1}}

\newcommand{\A}{\ensuremath{\adom{A}}\xspace}
\newcommand{\modA}{\ensuremath{\adom{A}_\mathsf{M}}\xspace}
\newcommand{\Absel}{\ensuremath{\ael{a}}\xspace}
\newcommand{\modAbsel}{\ensuremath{\ael{a}_\mathsf{m}}\xspace}

\newcommand{\relsem}[1]{\semantica{R}{#1}}

 % Proiezione/dropping

\newcommand{\lascia}{\ensuremath{\delta}}
\newcommand{\tieni}{\ensuremath{\pi}}

%\newcommand{\Absless}{\ensuremath{\sqsubseteq^\mathsf{a}}}
\newcommand{\modAbsless}{\ensuremath{\abs{\sqsubseteq}_\mathsf{m}}}

%\newcommand{\Absjoin}{\ensuremath{\sqcap^\mathsf{a}}}
\newcommand{\modAbsjoin}{\ensuremath{\abs{\sqcap}_\mathsf{m}}}

\newcommand{\aconc}{\ensuremath{\abs{\odot}}\xspace}


 % Operazioni sul dominio di vincoli
\newcommand{\cless}{\ensuremath{\preceq}}
\newcommand{\cgreater}{\ensuremath{\succeq}}
\newcommand{\ctop}{\ensuremath{\vernal}}
\newcommand{\cbot}{\ensuremath{\bot^\con}}
\newcommand{\cmeet}{\ensuremath{\curlywedge}}
\newcommand{\cjoin}{\ensuremath{\curlyvee}}
\newcommand{\cwiden}{\ensuremath{\hbox{\hbox to 0pt{\raisebox{4pt}{$\relbar$}}$\curlyvee$}}}
\newcommand{\bigcjoin}{\ensuremath{\bigcurlyvee}}

\newcommand{\alg}{\ensuremath{\mathcal{A}}}

 % Definizione di BodyOf
\newcommand{\bodyof}[1]{\ensuremath{\mathrm{bodyOf}(\mathbf{#1})}}

 % Definitione di tiedVars
\newcommand{\tiedvars}{\ensuremath{\mathrm{tiedVars}}}

  % Operazioni su Tracce
%\newcommand{\tracce}[1]{\ensuremath{\wp(#1^\infty)}\xspace}
\newcommand{\iniziatracce}[2]{\ensuremath{\mathcal{T}(#1,#2)}}
\newcommand{\giunzione}{\ensuremath{\frown}\xspace}
\newcommand{\conc}{\ensuremath{\odot}\xspace}
\newcommand{\last}{\ensuremath{\mathrm{last}}}

  % Funzioni Semantiche ``Astratte''
\newcommand{\flow}[1]{\ensuremath{\mathrm{flow}_\mathtt{#1}}}
\newcommand{\aflow}[1]{\ensuremath{\overline{\mathrm{flow}}_\mathtt{#1}}}

  % Definizione di ``approssima'', i.e. un vincolo approssima la semantica di un metodo
\newcommand{\approssima}{\ensuremath{\Subset}}

  % Approssimazione con vincoli della semantica di un metodo
\newcommand{\conmetodo}[5]{\ensuremath{\con_{#1}[\mathtt{#2}, \allowbreak \mathtt{#3}, \allowbreak \mathtt{#4}, \allowbreak \mathtt{#5}]}}



% Definizione del dominio degli osservabili
\newcommand{\osservabile}[2]{\ensuremath{\mathcal{O}_{#2}(\classe{#1})}}
\newcommand{\oless}{\ensuremath{{\sqsubseteq}_{o}}}
\newcommand{\osup}{\ensuremath{{\sqsupseteq}_{o}}}
\newcommand{\ojoin}{\ensuremath{{\sqcup}_{o}}}
\newcommand{\omeet}{\ensuremath{{\sqcap}_{o}}}
\newcommand{\otp}{\ensuremath{{\top}_{o}}}
\newcommand{\obot}{\ensuremath{{\bot}_{o}}}
\newcommand{\obigjoin}{{{\bigsqcup}{}}_{o}}

\newcommand{\domPiuPreciso}{\ensuremath{O[\parti{\Sigma}]}}

% Interfaccia di una classe
\newcommand{\interfaccia}[1]{\ensuremath{\iota(\classe{#1})}}


% Definizione dell' insieme delle astrazioni di un dominio +
% operazioni

\newcommand{\astrazioni}[1]{\ensuremath{\mathcal{A}(#1)}}
\newcommand{\dless}{\ensuremath{\leq}}
\newcommand{\djoin}{\ensuremath{\vee}}
\newcommand{\dmeet}{\ensuremath{\wedge}}

% Dominio dei valori
\newcommand{\dominioVal}[1]{\ensuremath{\mathit{#1}}}

% Stato interno ad un oggetto
\newcommand{\stato}{\ensuremath{\sigma}}

% Semantica backward concreta ed astratta
\newcommand{\backSemmetodo}[1]{\semantica{{M}^{<}}{#1}}
\newcommand{\abacksemmetodo}[1]{\semantica{\bar{M}^{<}}{#1}}


% Complexity
\newcommand{\Order}[1]{\ensuremath{\mathcal{O}(#1)}}

% Keywords

\newcommand{\Clousot}{\ensuremath{\mathtt{Clousot}}}

\newcommand{\code}[1]{\ensuremath{\mathtt{#1}}}

\newcommand{\takes}{\ensuremath{\leftarrow}}

\newcommand{\linea}[1]{\ensuremath{#1}}



% Syntax

\newcommand{\Assert}{\ensuremath{\mathtt{assert}}~}
\newcommand{\Assume}{\ensuremath{\mathtt{assume}}~}
\newcommand{\While}{\code{while}}
\newcommand{\If}{\code{if}}
\newcommand{\Else}{\code{else}}
\newcommand{\Skip}{\code{skip}}

\newcommand{\Stm}{\code{Stm}}
\newcommand{\Var}{\code{Var}}
\newcommand{\Exp}{\code{Exp}}
\newcommand{\BExp}{\code{BExp}}
\newcommand{\Bexp}{\BExp}
\newcommand{\ExpTwoOps}{\code{ExpTwoOps}}
\newcommand{\Lit}{\code{Lit}}
\newcommand{\Op}{\code{op}}
\newcommand{\Relop}{\code{relop}}
\newcommand{\Int}{\code{int}}

\newcommand{\Code}{\code{IstrStream}}
\newcommand{\Label}{\code{Label}}
\newcommand{\Istr}{\code{Istr}}
\newcommand{\Jump}{\code{jmp}}
\newcommand{\JumpIfTrue}{\code{jmpIf}}
\newcommand{\JumpIf}{\JumpIfTrue}
\newcommand{\Nop}{\code{nop}}

\newcommand{\Comp}{\ensuremath{\mathcal{C}}}
\newcommand{\Compexp}{\ensuremath{\mathcal{C}_{e}}}
\newcommand{\CompExp}{\Compexp}

\newcommand{\assign}{\el{assign}}
\newcommand{\test}{\el{test}}
\newcommand{\guard}{\test}
\newcommand{\checkif}{\el{check}}
\newcommand{\checkIf}{\checkif}

\newcommand{\aassign}{\el{\assign}}
\newcommand{\atest}{\el{\test}}

\newcommand{\arange}{\el{\mathsf{range}}}

\newcommand{\widening}{\ensuremath{\triangledown}}
\newcommand{\awidening}{\widening}
\newcommand{\narrowing}{\ensuremath{\vartriangle}}

\newcommand{\HLsem}[1]{\asemantica{H}{#1}}
\newcommand{\HLSem}[1]{\HLsem{#1}}
\newcommand{\LLsem}[1]{\asemantica{L}{#1}}
\newcommand{\LLSem}[1]{\LLsem{#1}}

\newcommand{\BigLLSem}[1]{\asemanticaBig{L}{#1}}
\newcommand{\bigLLSem}[1]{\BigLLSem{#1}}
\newcommand{\bigLLsem}[1]{\BigLLSem{#1}}

\newcommand{\result}{\code{res}}

% Definizioni per Subpolyhedra


\newcommand{\VarProg}{\ensuremath{\Var_\mathtt{P}}}
\newcommand{\VarSlack}{\ensuremath{\Var_\mathtt{S}}}

\newcommand{\variable}[1]{\ensuremath{\mathtt{#1}}}

\newcommand{\var}{\variable{v}}

\newcommand{\progvar}{\variable{x}}
\newcommand{\progvariable}[1]{\ensuremath{\progvar_{#1}}}

\newcommand{\slackvariable}[1]{\ensuremath{\beta_{#1}}}
\newcommand{\slackvar}{\slackvariable{}}

\newcommand{\slackvariableinfo}[1]{\ensuremath{\code{info}(#1)}}
\newcommand{\slackvarinfo}{\slackvariableinfo{\slackvariable{}}}


\newcommand{\reduction}[1]{\ensuremath{\rho(#1)}}
\newcommand{\simplify}[1]{\ensuremath{\sigma(#1)}}

\newcommand{\bottomS}{\ensuremath{\bottom_{S}}}
\newcommand{\topS}{\ensuremath{\top_{S}}}
\newcommand{\lessS}{\ensuremath{\aless_{S}}}
\newcommand{\joinS}{\ensuremath{\join_{S}}}
\newcommand{\wideningS}{\ensuremath{\widening_{S}}}

\newcommand{\intv}{\ael{i}}
\newcommand{\lineq}{\ael{l}}

\newcommand{\subpoly}{\ael{s}}

\newcommand{\subpolyPair}[2]{\ensuremath{\tupla{\lineq_{#1}^{#2};\ \intv_{#1}^{#2}}}}

%% Subsection trick to save some vertical space
\newcommand{\Fsubsubsection}[1]{\smallskip\noindent\textbf{#1}}


\newcommand{\Foxtrot}{\code{Foxtrot}}
\newcommand{\NET}{\code{.Net}}


\newcommand{\hint}[1]{\ensuremath{\mathbbm{h}_{#1}}}
\newcommand{\op}{\ensuremath{\diamond}}

\newcommand{\hintcomp}{\ensuremath{\Xi}}